\documentclass[10pt,onecolumn]{witseiepaper}
\usepackage{KJN}
\usepackage{graphicx}
\usepackage{url}

\newcommand{\ttt}{\texttt}

\title{ELEN4010 - PROJECT TESTING}

\author{Kayla-Jade Butkow (714227), Jared Ping (704447), Lara Timm (704157) \& Matthew van Rooyen (706612)}
\thanks{School of Electrical \& Information Engineering, University of the
Witwatersrand, Private Bag 3, 2050, Johannesburg, South Africa}


\begin{document}
\maketitle
\pagestyle{plain}
\setcounter{page}{1}

%%%%%%%%%%%%%%%%%%%%%%%%%%%%%%%%%%%%%%%%%%%%%%%%%%%%%%%%%%%%%%%%%%%%%%%%%%%%%%%%%%%%%%5
\section{Currently implemented unit tests and their corresponding issue numbers}

\textbf{28: Configure test and lint frameworks}

\begin{enumerate}
	\item Hello world test to ensure testing is working correctly
\end{enumerate}

\textbf{30: Implement logic to add item name to shopping list}

\begin{enumerate}
	\item Test that the size of empty shopping list is 0
	\item Ensure that the size of the shopping list increases when an item is added
	\item Test that the name of the added item is equal to that of the one appearing in the shopping list
\end{enumerate}

\textbf{32: Implement logic to add category/aisle}

\begin{enumerate}
	\item Test that the size of an empty category list is equal to 0
	\item Ensure that the size of the category list is increased when an item category is added
	\item Test that the name of the added item category is equal to that of the one appearing in the category list
\end{enumerate}

\textbf{34: Implement logic to enter multiple items onto a shopping list}

\begin{enumerate}
	\item Test that multiple items can be added to the shopping list
\end{enumerate}

\vspace{5mm}
\section{Currently implemented integration tests and their corresponding issue numbers}

\textbf{33: Render item's category/aisle on shopping list}

\begin{enumerate}
	\item Allow user to add and render a category to an item
	\item Default category appears as "Category" if not selected
\end{enumerate}

\textbf{35: Render multiple items on shopping list}

\begin{enumerate}
	\item Allow user to add and render multiple items to a list
\end{enumerate}

\textbf{81: Allow user to store edited name in database and render on website}

\begin{enumerate}
	\item test that item name can be added correctly
\end{enumerate}

\textbf{82: Allow user to store edited category/aisle in database and render on website}

\begin{enumerate}
	\item Allow user to edit category on list
	\item Edited category can be stored in the database and rendered
\end{enumerate}

\textbf{100: Users can tick off completed item cards which are saved in database}

\begin{enumerate}
	\item User is able to check off an item
	\item User is able to check off multiple items 
	\item The purchase status of one item is stored in the database and can be retrieved and rendered correctly 
	\item The purchase status of multiple items is stored in the database and can be retrieved and rendered correctly
\end{enumerate}

\textbf{104: Item quantities can added to items which is stored in the database and rendered on the website}

\begin{enumerate}
	\item Quantity can be added for a single item
	\item Quantity can be added for multiple items
	\item Quantity of one item can be stored in the database and can be retrieved and rendered correctly
	\item Quantity of multiple items can be stored in the database and can be retrieved and rendered correctly
\end{enumerate}

%\textbf{97: Site interaction flow for users}
%
%No new tests added but code to remove the implemented overlay from the screen was implemented to the tests to enable existing tests to pass

\textbf{98: User can create additional shopping list that is stored in database and rendered on website}

\begin{enumerate}
	\item Two shopping lists are created and the loading of the first list is tested
	\item Additional items can be added to the loaded shopping list
	\item Test that additional items can be added to the loaded shopping list and these items are stored in the database and can be retrieved and rendered correctly
	\item Test that item can be edited in a loaded shopping list and the edits are stored in the database and can be retrieved and rendered correctly
\end{enumerate}

\textbf{103: Item card can be deleted with changes saved in database and rendered on website}
\begin{enumerate}
	\item Item can be added to the item array and subsequently deleted
	\item Multiple items can be added to the item array and subsequently deleted
	\item A single item can be added and deleted, these changes are stored in the database and can be retrieved and rendered correctly
	\item Multiple items can be added and deleted, these changes are stored in the database and can be retrieved and rendered correctly
\end{enumerate}

\textbf{102: Lists will be sorted based on item completion}

\begin{enumerate}
	\item Multiple items can be added, one item is marked as purchased, the items are sorted by purchase status and the results are stored in the array and rendered correctly
\end{enumerate}

\textbf{96: Shopping list can be shared with someone via send grid email api}
\begin{enumerate}
	\item List can be shared with one user and the list of shared users is checked
	\item List can be shared with one user and their email is stored in the database and can be retrieved and rendered correctly
	\item List can be shared with multiple users and the list of shared users is checked
	\item List can be shared with multiple users and both emails are tested to be stored in the database and can be retrieved and rendered correctly
\end{enumerate}

\textbf{121: Shared shopping list access can be revoked}

\begin{enumerate}
	\item List is shared with two users, the first email is removed. Test that upon reload only the single correct email is loaded
\end{enumerate}

%\textbf{116: User can use email template to share their shopping list}
%
%\begin{enumerate}
%	\item 
%\end{enumerate}

\textbf{93: Allow user to add shopping list name, store in database and render on website}

\begin{enumerate}
	\item A shopping list name can be created and is rendered
	\item A shopping list name can be added and is stored in the database and can be retrieved and rendered correctly
	\item The shopping list name can be edited and is rendered
	\item The shopping list name can be edited and is stored in the database and can be retrieved and rendered correctly
	\item  A shopping list name can be added, as well as additional items, and the changes are stored in the database and can be retrieved and rendered correctly
\end{enumerate}

\textbf{115: Allow user to edit item quantity, store in database and render on website}

\begin{enumerate}
	\item Allow user to edit an item quantity
	\item Edited item quantity is persistent and can be rendered
	\item Multiple item quantities can be edited with the changes stored in the database and rendered
\end{enumerate}

\textbf{117: Allow the user to add notes about their shopping list, store them and render them}

\begin{enumerate}
	\item Shopping list notes can be added and are rendered correctly
	\item Shopping list noted can be added and are stored in the database and can be retrieved and rendered correctly
	\item Shopping list notes can be edited and are rendered correctly
	\item Shopping list notes can be edited and are stored stored in the database and can be retrieved and rendered correctly
\end{enumerate}

\textbf{122: Item category is chosen from set of defaults}

\begin{enumerate}
	\item Item category can be selected from the drop-down menu, the results are stored in the database and can be retrieved and rendered correctly
	\item Item category can be edited from the drop-down menu on its respective card and the changes are stored in the database and can be retrieved and rendered correctly
\end{enumerate}

\textbf{123: Sort item cards on shopping list by category (alphabetical)}

\begin{enumerate}
	\item Multiple items can be added by choosing a category from the drop-down, these items can be sorted alphabetically by category.
\end{enumerate}

\textbf{120: Shopping list colour changes are saved for when accessed at a later point}

\begin{enumerate}
	\item Card colour can be assigned to individual cards and is rendered correctly
	\item List colour can be assigned and is rendered correctly
	\item Card colours are stored in the database and can be retrieved and rendered correctly
\end{enumerate}

\end{document}
