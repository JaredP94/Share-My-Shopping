\documentclass[10pt,onecolumn]{witseiepaper}
\usepackage{KJN}
\usepackage{graphicx}
\usepackage{url}

\newcommand{\ttt}{\texttt}

\title{ELEN4010 - Sprint Retrospectives}

\author{}
\thanks{School of Electrical \& Information Engineering, University of the
Witwatersrand, Private Bag 3, 2050, Johannesburg, South Africa}


\begin{document}

\maketitle
\pagestyle{plain}
\setcounter{page}{1}

\section*{SPRINT 1}
\subsection*{Attendees:}
Kayla-Jade Butkow, Jared Ping, Lara Timm, Matt van Rooyen
\subsection*{Agenda:} 
To critically analyse the sprint. Calculate the sprint velocity.

\subsection*{Minutes of meeting:}
Start date: 24/04/2018 \\
Start time: 19:00

\subsubsection*{What went right:}
\begin{enumerate}
	\item We met our expected deadlines, both individually and as a team
	\item Collaborative work was performed smoothly
	\item All developer stories assigned to the sprint milestone were completed
	\item An efficient code review process was created and maintained for the duration of the sprint
	\item Minimal variation between developers 
	\item The use of daily stand-ups to track sprint progress
\end{enumerate}

\subsubsection*{What went wrong:}
\begin{enumerate}
	\item The collaborative components of the agile development process resulted in unexpected delays
	\item Logic based pull requests were opened without any tests
	\item A branch was merged before suggested changes were made
	\item Poor communication when developer required assistance
	\item Only unit tests were implemented 
	\item Unsuccessful configuration of coveralls
\end{enumerate}

\subsubsection*{What needs improvement:}
\begin{enumerate}
	\item Weekly meeting to improve scrum tactics. This will ensure that all team members are well versed in agile techniques to accommodate a more streamlined process
	\item A minimum of one tests must be implemented for each logic based developer story 
	\item Implementation of tests from all levels of the test pyramid
	\item More rigorous pull request reviews. Any changes should be required, rather than suggested
	\item Improve structure of daily stand-ups to ensure that assistance is readily available 
\end{enumerate}

\subsubsection*{Velocity:}
Each developer story was assigned an effort rating between 1 and 10. Using these efforts, the sprint velocity was calculated to be 49. The ideal velocity for the next sprint is therefore 49.
\newpage
\section*{SPRINT 2}
\subsection*{Attendees:}
Kayla-Jade Butkow, Jared Ping, Lara Timm, Matt van Rooyen

\subsection*{Agenda:} 
To critically analyse the sprint and calculate the sprint velocity.

\subsection*{Minutes of meeting:}
Start date: 01/05/2018 \\
Start time: 19:00

\subsubsection*{What went right:}
\begin{enumerate}
	\item All developer stories assigned to the sprint were completed
	\item We met our expected deadlines, both individually and as a team
	\item Collaborative work was performed smoothly
	\item Minimal variation between developers 
	\item The use of daily stand-ups to track sprint progress
	\item Current sprint velocity is very close to previous velocity
	\item Choice to set up local databases has (and will in future) sped up the time taken for database related operations to be tested
	\item The wiki for the repository is being well utilised by the team and has assisted in remedying issues and streamlining processes
\end{enumerate}

\subsubsection*{What went wrong:}
\begin{enumerate}
	\item Violation of trunk based development regarding a single merge into master. Pull request was not opened, and as such there was no opportunity for the team to review. These changes were reverted, and the correct process was followed. However, for a short period of time, unmoderated changes were deployed.
	\item Lots of time was consumed by setting up local databases
	\item Unsuccessful configuration of coveralls
	\item Issues with configuring web service to connect to database
	\item Sprint board required further iteration to cater for correctly sliced developer stories
\end{enumerate}

\subsubsection*{What needs improvement:}
\begin{enumerate}
	\item Selenium needs to be implemented to allow for integration and acceptance tests
	\item Code coverage must be tested using Coveralls
	\item A development deployment environment should be used for pre-release testing. This will prevent production downtime
	\item Comprehensive tests required to adhere to the testing pyramid
\end{enumerate}

\subsubsection*{Velocity:}
The velocity for this sprint was 47. 

\newpage
\section*{SPRINT 3}

\subsection*{Attendees:}
Kayla-Jade Butkow, Jared Ping, Lara Timm, Matt van Rooyen

\subsection*{Agenda:} 
To critically analyse the sprint and calculate the sprint velocity.

\subsection*{Minutes of meeting:}
Start date: 08/05/2018 \\
Start time: 19:00

\subsubsection*{What went right:}
\begin{enumerate}
	\item All developer stories assigned to the sprint were completed
	\item We met our expected deadlines, both individually and as a team
	\item Integration and acceptance tests were implemented
	\item No violations of trunk based development 
	\item Optimised database schema has been implemented. This improves data access overhead
	
\end{enumerate}

\subsubsection*{What went wrong:}
\begin{enumerate}
	\item Unexpected delays in pull request reviews resulting in delays in feature deployment
	\item Inaccurate effort levels and durations were assigned to the developer stories in this sprint. This led to inadequate amounts of time for testing 
	\item Unable to set up and configure Coveralls
\end{enumerate}

\subsubsection*{What needs improvement:}
\begin{enumerate}
	\item Estimation of effort and duration for developer stories needs to be made more accurate
	\item Implement a morning and afternoon stand-up. The morning stand-up is to discuss what you are working on during that day. The afternoon stand-up will be used for fixing any issues and reviewing outstanding pull requests 
\end{enumerate}

\subsubsection*{Velocity:}
The velocity for this sprint was 50. 

\newpage
\section*{SPRINT 4}

\subsection*{Attendees:}
Kayla-Jade Butkow, Jared Ping, Lara Timm, Matt van Rooyen

\subsection*{Agenda:} 
To critically analyse the sprint and calculate the sprint velocity.

\subsection*{Minutes of meeting:}
Start date: 16/05/2018 \\
Start time: 19:00

\subsubsection*{What went right:}
\begin{enumerate}
	\item All developer stories for the sprint were completed
	\item All developer stories set out for the project were completed
	\item Added integration and acceptance tests for each developer story that was implemented
	\item We met our expected deadlines, both individually and as a team
	
\end{enumerate}

\subsubsection*{What went wrong:}
\begin{enumerate}
	\item One pull request was merged before the Travis tests had passed. As such, there is a failing Travis test on one of the branches that cannot be fixed
	\item Anomalies when running Selenium tests were found. These resulted in tests failing randomly and for no reason. This resulted in delays in feature deployment
	\item Unable to track code coverage
\end{enumerate}

\subsubsection*{What needs improvement:}
\begin{enumerate}
	\item Quality of test coverage
	\item Developers need to better adhere to the code guidelines set out by the group. This will allow for a quicker review process
\end{enumerate}

\subsubsection*{Velocity:}
The velocity for this sprint was 50. 
\end{document}