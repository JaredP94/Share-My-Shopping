\documentclass[10pt,onecolumn]{witseiepaper}
\usepackage{KJN}
\usepackage{listings} % for code listings
\usepackage[dvipsnames]{xcolor} % for the listings
\usepackage{verbatim}
\usepackage{courier} % for the courier font for the code listings

\newcommand{\ttt}{\texttt}

\definecolor{lightgray}{rgb}{.9,.9,.9}
\definecolor{darkgray}{rgb}{.4,.4,.4}
\definecolor{purple}{rgb}{0.65, 0.12, 0.82}

\lstdefinelanguage{JavaScript}{
	keywords={typeof, new, true, false, catch, function, return, null, catch, switch, var, if, in, while, do, else, case, break, let},
	keywordstyle=\color{NavyBlue}\bfseries,
	ndkeywords={class, export, boolean, throw, implements, import, this},
	ndkeywordstyle=\color{darkgray}\bfseries,
	identifierstyle=\color{black},
	sensitive=false,
	comment=[l]{//},
	morecomment=[s]{/*}{*/},
	commentstyle=\color{purple}\ttfamily,
	stringstyle=\color{BrickRed}\ttfamily,
	morestring=[b]',
	morestring=[b]"
}


\title{ELEN4010 - CODE STYLING GUIDE}

\author{Kayla-Jade Butkow (714227), Jared Ping (704447), Lara Timm (704157) \& Matthew van Rooyen (706612)}
\thanks{School of Electrical \& Information Engineering, University of the
Witwatersrand, Private Bag 3, 2050, Johannesburg, South Africa}


\begin{document}

\maketitle
\pagestyle{plain}
\setcounter{page}{1}

\section{INTRODUCTION}

\section{JAVASCRIPT}
\lstset{language=JavaScript, basicstyle=\ttfamily}%, frame=single}

\subsection{\textbf{Variables}}

Use \texttt{camelCase}  for names of variables and functions. All names should start with a lower-case letter. Acronyms should be upper-case.

\begin{lstlisting}[backgroundcolor=\color{lightgray!60}]
firstName = "John";
lastName = "Doe";

index = 0;
counterValue = 0;

elementHTML = document.getElementById("button_1").innerHTML
\end{lstlisting}

Prefer declaring variables with \texttt{let} over \texttt{var} for better scoping when appropriate. 

\begin{lstlisting}[backgroundcolor=\color{lightgray!60}]
for (let i = 0; i < 10; i++) {
	doSomthing();
}
\end{lstlisting}

\subsection{\textbf{Spacing}}

Do not put spaces around unary operators or before a comma or semi-colon.

\begin{lstlisting}[backgroundcolor=\color{lightgray!60}]
i++;
f(x, y);
\end{lstlisting}

Always put spaces around binary (assignment, equality and boolean) and ternary (arithmetic) operators. Always place a space after a comma.

\begin{lstlisting}[backgroundcolor=\color{lightgray!60}]
y = m * x + c;
f(x, y);
c = !a | b;
\end{lstlisting}

Tabs are preferred over spaces.

\begin{lstlisting}[backgroundcolor=\color{lightgray!60}]
function toCelcius(farenheit) {
    return (5 / 9) * (fahrenheit - 32);
}
\end{lstlisting}

Place spaces between control statements and their parentheses. Do not place spaces between a function and its parentheses, or between a parenthesis and its content.

\begin{lstlisting}[backgroundcolor=\color{lightgray!60}]
for (let i = 0; i < 10; i++) {
    doSomthing();
}
\end{lstlisting}

\subsection{\textbf{Line Breaking \& Braces}}

Each statement should get its own line.

\begin{lstlisting}[backgroundcolor=\color{lightgray!60}]
i++;
j++;
if (boolCondition)
    doSomthing();
\end{lstlisting}

An \texttt{else} statement should go on the same as a preceding close brace if one is present, otherwise should line up with the \texttt{if} statement.

Open braces should be placed should be placed on the same line as the preceding code block and close braces should be placed on their own line.

One-line control clauses should not use braces (unless comments are included) or a single statement spans multiple lines.

\begin{lstlisting}[backgroundcolor=\color{lightgray!60}]
if (boolCondition) {
    ...
} else {
    someFunction(veryLongParam1, veryLongParam2, ...
    veryLongParam5);
}

if (boolCondition)
    doSomething();
else
    doADifferentThing();
    
if (boolCondition)
    doSomething();
else {
    ...
}    
\end{lstlisting}

\subsection{\textbf{Strings}}

Prefer double quotes over single quotes for strings.

\begin{lstlisting}[backgroundcolor=\color{lightgray!60}, showstringspaces=false]
console.log("This is how strings should be declared");
\end{lstlisting}

Use implicit line joining for multi-line strings.

\begin{lstlisting}[backgroundcolor=\color{lightgray!60}, showstringspaces=false]
console.log("This is how strings should be declared\n"
            "It's really quite nice to look at\n);
\end{lstlisting}


\section{HTML}
\lstset{language=HTML, basicstyle=\ttfamily}%, frame=single}

\subsection{\textbf{Nesting and Indentation}}

When tags contain other tags in their body, the nested tags should be placed on a new line. All new levels of nesting should be indented by another level.

\begin{lstlisting}[backgroundcolor=\color{lightgray!60}]
<html>
  <head>
    <title>Example of good nesting and indentation</title>
  </head>
</html>
\end{lstlisting}

%\begin{lstlisting}[backgroundcolor=\color{red!20}]
%<html><head><title>Example of bad nesting</title></head></html>
%
%<html>
%<head>
%<title>Example of bad indentation</title>
%</head>
%</html>
%\end{lstlisting}

\subsection{\textbf{Tags \& Attributes}}

Tag names should all be lower-case.

\begin{lstlisting}[backgroundcolor=\color{lightgray!60}]
<div>
  <p>Lower-case tags are better</p>
</div>
\end{lstlisting}

%\begin{lstlisting}[backgroundcolor=\color{red!20}]
%<DIV>
%<P>Uppercase tags are not ideal</p>
%</Div>
%\end{lstlisting}

All tags should be closed unless they are non-container tags.

\begin{lstlisting}[backgroundcolor=\color{lightgray!60}]
<div>Container tags</div>
<p>must be closed</p>
<br>This doesnt have to be closed
\end{lstlisting}

%\begin{lstlisting}[backgroundcolor=\color{red!20}]
%<div>Example of bad 
%<p>HTML style
%<br>Closing this is unnecessary</br>
%\end{lstlisting}

Use lower-case attribute names.

\begin{lstlisting}[backgroundcolor=\color{lightgray!60}]
<div class="dropdown">Dropdown menu</div>
\end{lstlisting}



\end{document}
